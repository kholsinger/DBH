\documentclass[12pt]{article}
\usepackage{bm}

\newcommand{\gi}{{\tt gi}}
\newcommand{\dbh}{{\tt dbh}}

\begin{document}

\section*{Where we're headed}

We have two sets of data: (1) annual growth increment (\gi) data paired
with monthly precipitation and temperature data and (2) diameter at
breast height (\dbh) data (2 observations) paired with observations on stand
characteristics and local environmental variables. Ultimately, the
goal is to use couple the \gi\ data, which are available for fewer
individuals, with the \dbh\ data, which are available for more
individuals to produce refined. The model we ultimately want is
something like this:

\begin{eqnarray*}
y^{(dbh_2)}_{ij} &\sim& \mbox{N}(\mu_{ij}, \sigma^2_{dbh}) \\
{\bm y}^{(gi)}_{kl} &\sim& \mbox{N}({\bm\nu}_{kl}, \Sigma) \\
\mu_{ij} &=& y^{(dbh_1)}_{ij} + \exp{({\bm\beta X}_i + \alpha\phi_{l_j} +
             \epsilon_{S_i} + \epsilon_{j})} \\
\phi_{l_j} &=& {\bm\gamma Z}_{l_j\cdot} \\
\epsilon_{S_i} &\sim& \mbox{N}(0, \sigma^2_{species}) \\
\epsilon_{j} &\sim& \mbox{N}(0, \sigma^2_{plot^{(dbh)}}) \\
\nu_{kl} &=& \nu_k + \nu_l \\
\nu_l &=& {\bm\gamma Z}_{lm} \\
\nu_k &\sim& \mbox{N}(\nu^{(plot)}_{M_k}, \sigma^2_{indiv}) \\
\nu^{(plot)}_{M_k} &\sim& \mbox{N}(\nu_0, \sigma^2_{plot^{(gi)}})
\quad ,
\end{eqnarray*}
where $y_{ij}^{(dbh_2)}$ is the second measurement of diameter at
breast height of individual $i$ in plot $j$, $y_{ij}^{(dbh_1)}$ is the
first measurement of diameter breast height for individual $i$ in plot
$j$, ${\bm X}_i$ is the vector of of site covariates for individual
$i$ (including an intercept term), ${\bm Z}_{jm}$ is the vector of
weather data at site $l$ in year $m$, $i=1,\dots,I$ indexes individual
trees in the \dbh\ data, $j=1,\dots,J$ indexes sites, $k=1,\dots,K$
indexes individual trees in the \gi\ data, $l=1,\dots,L$ indexes plots
within sites, and $l_j$ refers to the site index in which plot $j$
occurs. I haven't included all of the necessary priors, but this
captures other features of the model.

This approach ``ties'' the \dbh\ and \gi\ data together through
$\bm\gamma$, the regression coefficients for monthly precipitation and
temperature data. The easiest way to think of this is that there is a
latent growth process that we observe yearly in the \gi\ data and only
over an interval in the \dbh\ data. Since growth in the latent process
is unobservable, we can only relate it to the observable data as a
ratio, $\alpha$, of \dbh\ growth to \gi\ growth. Notice that
$\alpha\phi_{l_j}$ is an adjustment to the intercept. As a result,
$\alpha$ in non-identifiable, meaning that we can't couple the models
this way, unless $L > 1$.

\section*{What we have}

In our case (so far) $L = 1$, so we can't fit that model. We can,
however, fit a one that couples the \gi\ and \dbh\ data in a different
way. Specifically, in the model above the prior for plot effect in the
\gi\ data is
\[
\nu^{(plot)}_{M_k} \sim \mbox{N}(\nu_0, \sigma^2_{plot^{(gi)}})
\quad .
\]
We also have an estimate of the plot effect from the \dbh\ data,
namely $\epsilon_j$. So instead of using the same prior mean, $\nu_0$
for all plot effects in the \gi\ data, we can use plot-specific prior
means related to the plot effects from the \dbh\ data.
\begin{eqnarray*}
\nu^{(plot)}_{M_k} &\sim& \mbox{N}(\nu_{0,{M_k}},
                          \sigma^2_{plot^{(gi)}}) \\
\nu_{0,{M_k}} &\propto& \exp(\epsilon_{M_k}) \quad ,
\end{eqnarray*}
where the constant of proportionality is equal to the number of years
between the first and second measurement of diameter at breast
height. An alternative approach would be
\begin{eqnarray*}
\nu^{(plot)}_{M_k} &=& \alpha^*\exp(\epsilon_{M_k}) \\
\alpha^* &\sim& \mbox{N}^+(0,\sigma^2_{plot^{(gi)}}) \quad .
\end{eqnarray*}
In the first approach, we might put a prior on
$\sigma^2_{plot^{(gi)}}$ and estimate it. In the second approach, we'd
probably pick a constant. $\sqrt 2$ would be a reasonable choice for a
constant, since it would produce roughly a 95\% chance of
$\alpha \in [0,1]$.

With either approach we can compare predicted values in \dbh\ and \gi\
with observed and determine whether the differences are smaller than
when the models are run independently. We can also look at the
credible intervals associated with both the regression coefficients
and the posterior predictions. It could be that the coupled model
leads to more precise predictions even if the mean posterior
prediction differs little between models.

\end{document}