\documentclass[12pt]{article}
\usepackage{bm}

\newcommand{\gi}{{\tt gi}}
\newcommand{\dbh}{{\tt dbh}}

\begin{document}

\section*{Where we're headed}

We have two sets of data: (1) annual growth increment (\gi) data paired
with monthly precipitation and temperature data and (2) diameter at
breast height (\dbh) data (2 observations) paired with observations on stand
characteristics and local environmental variables. Ultimately, the
goal is to use couple the \gi\ data, which are available for fewer
individuals, with the \dbh\ data, which are available for more
individuals to produce refined. The model we ultimately want is
something like this:

\begin{eqnarray*}
y^{(dbh_{inc})}_{ij} &=& y^{(dbh_2)}_{ij} - y^{(dbh_1)}_{ij} \\
y^{(dbh_{inc})}_{ij} &\sim& \mbox{N}(\mu_{ij}, \sigma^2_{dbh}) \\
{\bm y}^{(gi)}_{kl} &\sim& \mbox{N}({\bm\nu}_{kl}, \Sigma) \\
\mu_{ij} &=& {\bm\beta X}_{l_i}^{(j)} + \alpha\phi_{l_j} +
             \epsilon_{S_i} + \epsilon_{j} \\
\phi_{l_j} &=& {\bm\gamma Z}_{l_j\cdot} \\
\epsilon_{S_i} &\sim& \mbox{N}(0, \sigma^2_{species}) \\
\epsilon_{j} &\sim& \mbox{N}(0, \sigma^2_{plot^{(dbh)}}) \\
\nu_{kl} &=& \nu_k + \nu_l \\
\nu_l &=& {\bm\gamma Z}_{lm} \\
\nu_k &\sim& \mbox{N}(\nu^{(plot)}_{M_k}, \sigma^2_{indiv}) \\
\nu^{(plot)}_{M_k} &\sim& \mbox{N}(\nu_0, \sigma^2_{plot^{(gi)}})
\quad ,
\end{eqnarray*}
where $y_{ij}^{(dbh_2)}$ is the second measurement of diameter at
breast height of individual $i$ in site $j$, $y_{ij}^{(dbh_1)}$ is the
first measurement of diameter breast height for individual $i$ in site
$j$, ${\bm X}_{l_i}^{(j)}$ is the vector of of covariates for individual $i$
in plot $l$ within site $j$ (including an intercept term), ${\bm Z}_{lm}$ is the
vector of weather data at site $l$ in year $m$, $i=1,\dots,I$ indexes
individual trees in the \dbh\ data, $j=1,\dots,J$ indexes sites,
$k=1,\dots,K$ indexes individual trees in the \gi\ data, $l=1,\dots,L$
indexes plots within sites, and $l_j$ refers to the site index in
which plot $j$ occurs. I haven't included all of the necessary priors,
but this captures other features of the model.

This approach ``ties'' the \dbh\ and \gi\ data together through
$\bm\gamma$, the regression coefficients for monthly precipitation and
temperature data. The easiest way to think of this is that there is a
latent growth process that we observe yearly in the \gi\ data and only
over an interval in the \dbh\ data. Since growth in the latent process
is unobservable, we can only relate it to the observable data as a
ratio, $\alpha$, of \dbh\ growth to \gi\ growth. Notice that
$\alpha\phi_{l_j}$ is an adjustment to the intercept. As a result,
$\alpha$ in non-identifiable, meaning that we can't couple the models
this way, unless $L > 1$.

\section*{What we have}

In our case (so far) $L = 1$, so we can't fit that model. We can,
however, fit one that couples the \gi\ and \dbh\ data in a different
way. Specifically, we make the following changes:
\begin{eqnarray*}
\mu_{ij} &=& {\bm\beta X}_{l_i} + \epsilon_{S_i} + \epsilon_{j} \\
\nu_l &=& {\bm\gamma Z}_{lm} + \alpha{\bm\beta X}_l \quad .
\end{eqnarray*}
In other words, we couple the \gi\ and \dbh\ data through regression
coefficients on the local plot variables within the single site we
have. We won't be showing how parameters for monthly weather data from
\gi\ enhance prediction of \dbh\, but we'll show something similar by
showing how data shared between the two data sets can enhance
prediction. We do that by comparing the fits to the two data sets in
the model above with one with
\begin{eqnarray*}
\mu_{ij} &=& {\bm\beta_{\mu} X}_{l_i} + \epsilon_{S_i} + \epsilon_{j} \\
\nu_l &=& {\bm\gamma Z}_{lm} + {\bm\beta_{\nu} X}_l \quad .
\end{eqnarray*}
and independent normal priors on the $\bm\beta_{\mu}$ and $\bm\beta_{\nu}$.

The preliminary results (in {\tt model-comparison.txt}) show a mixed
result. The predictive performance of the coupled model (the first one
in this section) is strongly favored for \gi\, while the uncoupled
model is strongly favored for \dbh.

After getting these results, it occurred to me that there's a better
approach, which I'm exploring now. Instead of independent normal
priors on the $\bm\beta_{\mu}$ and $\bm\beta_{\nu}$, we use a
bivariate normal prior on each component of them and estimate the
correlation between the regression coefficients in the two
models. That will give us the best compromise between complete
independence (uncoupled) and perfect correlation (coupled).

\end{document}